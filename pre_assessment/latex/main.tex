\documentclass[letterpaper,12pt,addpoints]{exam}
\usepackage[utf8]{inputenc}
\usepackage[english]{babel}
\usepackage{fancyvrb}

\usepackage[top=1in, bottom=1in, left=0.75in, right=0.75in]{geometry}
\usepackage{amsmath,amssymb}
\setlength\FrameSep{4pt}
\newcommand{\university}{Massachussets Institute of Technology}
\newcommand{\faculty}{~}
\newcommand{\class}{Neuroscience of programming~}
\newcommand{\examnum}{Python fluency test}
\newcommand{\examdate}{~}
\newcommand{\content}{\textbf{Contact}: Anna Ivanova (annaiv@mit.edu), Shash Srikant (shash@mit.edu)}
\newcommand{\contact}{}
\newcommand{\timelimit}{60 minutes}
\newcommand{\filename}{0}

\usepackage{listings}

% Default fixed font does not support bold face
\DeclareFixedFont{\ttb}{T1}{txtt}{bx}{n}{12} % for bold
\DeclareFixedFont{\ttm}{T1}{txtt}{m}{n}{12}  % for normal

% Custom colors
\usepackage{color}
\definecolor{deepblue}{rgb}{0,0,0.5}
\definecolor{deepred}{rgb}{0.6,0,0}
\definecolor{deepgreen}{rgb}{0,0.5,0}
% Python style for highlighting
\newcommand\pythonstyle{\lstset{
		language=Python,
		basicstyle=\ttm,
		breaklines=true,
		postbreak=\mbox{\textcolor{red}{$\hookrightarrow$}\space},
		otherkeywords={self},             % Add keywords here
		keywordstyle=\ttb\color{deepblue},
		emph={MyClass,__init__},          % Custom highlighting
		emphstyle=\ttb\color{deepred},    % Custom highlighting style
		stringstyle=\color{deepgreen},
		frame=tb,                         % Any extra options here
		showstringspaces=false            % 
	}}
	

\newcommand\pythonstylelines{\lstset{
		language=Python,
		basicstyle=\ttm,
		breaklines=true,
		numbers=left,
		stepnumber=1,
		postbreak=\mbox{\textcolor{red}{$\hookrightarrow$}\space},
		otherkeywords={self},             % Add keywords here
		keywordstyle=\ttb\color{deepblue},
		emph={MyClass,__init__},          % Custom highlighting
		emphstyle=\ttb\color{deepred},    % Custom highlighting style
		stringstyle=\color{deepgreen},
		frame=tb,                         % Any extra options here
		showstringspaces=false            % 
	}}

	

\newcommand\pythonstylenoframe{\lstset{
		language=Python,
		basicstyle=\ttm,
		breaklines=true,
		postbreak=\mbox{\textcolor{red}{$\hookrightarrow$}\space},
		otherkeywords={self},             % Add keywords here
		keywordstyle=\ttb\color{deepblue},
		emph={MyClass,__init__},          % Custom highlighting
		emphstyle=\ttb\color{deepred},    % Custom highlighting style
		stringstyle=\color{deepgreen},
		%frame=tb,                         % Any extra options here
		showstringspaces=false            % 
	}}

	
% Python for external files
\newcommand\pythonexternal[2][]{{
		\pythonstyle
		\lstinputlisting[#1]{#2}}}


\newcommand\pythonexternallines[2][]{{
		\pythonstylelines
		\lstinputlisting[#1]{#2}}}


\newcommand\pythonexternalnoframe[2][]{{
		\pythonstylenoframe
		\lstinputlisting[#1]{#2}}}

\newcommand{\emptybox}[2][\textwidth]{%
	\begingroup
	\setlength{\fboxsep}{-\fboxrule}%
	\centering\noindent\framebox[#1]{\rule{0pt}{#2}}%
	\endgroup
}

\pagestyle{headandfoot}
\firstpageheader{}{}{}
\firstpagefooter{}{Page \thepage\ of \numpages}{}
\runningheader{\class}{\examnum}{\examdate}
\runningheadrule
\runningfooter{}{Page \thepage\ of \numpages}{}

\begin{document}
	
	\title{\Large \textbf{\university\\ \faculty\\
			\bigskip
			\class -- \examnum \\ }}
	\author{\textbf{Contact}: Anna Ivanova (annaiv@mit.edu), Shash Srikant (shash@mit.edu)}
	\date{\examdate}
	\maketitle
	\begin{flushleft}
		\makebox[12cm]{\textbf{Name}:\ \hrulefill}
		\medskip \medskip
		\newline \newline
		\makebox[12cm]{\textbf{E-mail ID}:\ \hrulefill}
		\newline *If you have an \texttt{@mit.edu} email ID, please fill in that.
	\end{flushleft}
	\noindent \rule{\textwidth}{1pt}
	
	\begin{itemize}
		\item This exam contains \numpages\ pages (including this cover page) and \numquestions\ questions. Unless specified, assume Python version $>$ 3.6.0. We do not however assess knowledge of specific libraries.
        \item Please answer within the box/space provided. Only those will be considered your final answers.
		\item There are three types of questions on this test. One, which asks you to write out the expected output of a program. Two, which asks you to fix an error in a program. Three, which asks you to design an algorithm and implement it in Python.
		\item For questions requiring you to fix the error, 
        \begin{itemize}
            \item you do not have to re-write the entire fixed program. Only rewrite lines containing your fixes. 
            \item there could be multiple ways to answer these questions. Answering any one is fine.
            \item do not replace the entire code snippet with an alternate approach. Try fixing the provided lines as much as possible.
        \end{itemize}
		\item All questions are weighted equally.
        \item If you're unable to answer any question but have an idea about it, write it down in the box provided. 
Likewise, if you think any question is ambiguous, write down your concern.
	\end{itemize}

Good luck and happy Pythoning!
	
\clearpage
\begin{questions}
	% guess the output items
    \question
What does the following code output?
\pythonexternal{codes/23.py}
\emptybox[0.95\textwidth]{2cm}

    \question
What does the following code output?
\pythonexternal{codes/24.py}
\emptybox[0.95\textwidth]{2cm}

    \question
What does the following code output?
\pythonexternal{codes/25.py}
\emptybox[0.95\textwidth]{2cm}

    \clearpage

    \question
What does the following code output?
\pythonexternal{codes/26.py}
\emptybox[0.95\textwidth]{2cm}

    \question
What does the following code output?
\pythonexternal{codes/27.py}
\emptybox[0.95\textwidth]{2cm}

    \question
What does the following code output?
\pythonexternal{codes/28.py}
\emptybox[0.95\textwidth]{2cm}

    \clearpage

    \question
What does the following code output?
\pythonexternal{codes/1.py}
\emptybox[0.95\textwidth]{2cm}
	\question
What does the following code output?
\pythonexternal{codes/2.py}
\emptybox[0.95\textwidth]{4cm}
	\question
What does the following code output?
\pythonexternal{codes/3.py}
\emptybox[0.95\textwidth]{3cm}
	\clearpage
	
	\question
What does the following code output?
\pythonexternal{codes/4.py}
\emptybox[0.95\textwidth]{3cm}
	\question
What does the following code output?
\pythonexternal{codes/6.py}
\emptybox[0.95\textwidth]{3cm}
	%\question
What does the following code output?
\pythonexternal{codes/8.py}
\emptybox[0.95\textwidth]{2cm}
	%\clearpage
	
	%\question
What does the following code output?
\pythonexternal{codes/7.py}
\emptybox[0.95\textwidth]{4cm}	
	\question
What does the following code output?
\pythonexternal{codes/10.py}
\emptybox[0.95\textwidth]{4cm}
	\clearpage
	
	\question
You are given two files - \texttt{one.py} and \texttt{two.py}. \texttt{two.py} is in the same directory as \texttt{one.py} and imports it. You run \texttt{two.py}. What output do you see?
\pythonexternal{codes/5.py}
\emptybox[0.95\textwidth]{8cm}
	\clearpage
	
	\question
What does the following code output?
\pythonexternal{codes/9.py}
\emptybox[0.95\textwidth]{4cm}
	
	% debugging items
	\question
\renewcommand{\filename}{11} 
While implementing the following script,  you see the error mentioned below. Fix the error. Rewrite code in the empty space on the right.
\noindent
\pythonexternalnoframe{codes/\filename b.py}
\begin{minipage}{.45\textwidth}
	\pythonexternallines{codes/\filename a.py}
\end{minipage}
\begin{minipage}{.45\textwidth}
		{\phantom{\pythonexternalnoframe{codes/\filename a.py}}}
\end{minipage}

	\clearpage
	
	\question
\renewcommand{\filename}{12} 
While implementing the following script,  you see the error mentioned below. Fix the error. Rewrite code in the empty space on the right.
\noindent
\pythonexternalnoframe{codes/\filename b.py}
\begin{minipage}{.45\textwidth}
	\pythonexternallines{codes/\filename a.py}
\end{minipage}
\begin{minipage}{.45\textwidth}
		{\phantom{\pythonexternalnoframe{codes/\filename a.py}}}
\end{minipage}

	\vspace{3cm}
	\question
\renewcommand{\filename}{13} 
Your friend has written a simple game which accepts integers between 1 and 10. Running her script generates the following error. Fix the error. Rewrite code in the empty space on the right.
\noindent
\pythonexternalnoframe{codes/\filename b.py}
\begin{minipage}{.55\textwidth}
	\pythonexternallines{codes/\filename a.py}
\end{minipage}
\begin{minipage}{.45\textwidth}
		{\phantom{\pythonexternalnoframe{codes/\filename a.py}}}
\end{minipage}
	\clearpage
	
	\question
\renewcommand{\filename}{15} 
While implementing the following script,  you see the error mentioned below. Fix the error. Rewrite code in the empty space on the right.
\noindent
\pythonexternalnoframe{codes/\filename b.py}
\begin{minipage}{.55\textwidth}
	\pythonexternallines{codes/\filename a.py}
\end{minipage}
\begin{minipage}{.45\textwidth}
		{\phantom{\pythonexternalnoframe{codes/\filename a.py}}}
\end{minipage}

	\vspace{0.5cm}
	\question
\renewcommand{\filename}{16} 
You have two files - \texttt{base.py} and \texttt{inherit.py}. Running \texttt{inherit.py} results in the following error. Fix the error. Rewrite code in the empty space on the right.
\noindent
\pythonexternalnoframe{codes/\filename b.py}
\begin{minipage}{.55\textwidth}
	\pythonexternallines{codes/\filename a.py}
\end{minipage}
\begin{minipage}{.45\textwidth}
	{\phantom{\pythonexternalnoframe{codes/\filename a.py}}}
\end{minipage}
	\clearpage
	
	\question
\renewcommand{\filename}{17} 
Alice writes a method to calculate the lowest positive number in a list of integers. On running it on a simple test case like \texttt{[16, -6, 1, 6, 6]}, she sees the following error. Fix the error. Rewrite code in the empty space on the right.
\noindent
\pythonexternalnoframe{codes/\filename b.py}
\begin{minipage}{.55\textwidth}
	\pythonexternallines{codes/\filename a.py}
\end{minipage}
\begin{minipage}{.45\textwidth}
	{\phantom{\pythonexternalnoframe{codes/\filename a.py}}}
\end{minipage}
	\clearpage
	
	%\question
\renewcommand{\filename}{18} 
Bob writes a class to check if he's able to control the visibility of a list of secrets defined in a parent class. He defines one class \texttt{F} and inherits it in classes \texttt{F\_some} - which explicitly accesses the secret, and \texttt{F\_none} - which does not.
Unfortunately, in his implementation, he finds that the inherited class \texttt{F\_none} is able to access this secret! This is the output on running the code 

\pythonexternalnoframe{codes/\filename b.py}

Help Bob fix the inheritance so that \texttt{F\_none} is not able to access the secret. Do not change the implementation of class \texttt{F}.

\begin{minipage}{.55\textwidth}
	\pythonexternallines{codes/\filename a.py}
\end{minipage}
\begin{minipage}{.45\textwidth}
	{\phantom{\pythonexternalnoframe{codes/\filename a.py}}}
\end{minipage}
	%\clearpage
	
	\question
\renewcommand{\filename}{19} 
You want to design a class in a way that it can dynamically call one of \texttt{function1} or \texttt{function2}. You have an initial attempt at making it work, but it compiles to the error mentioned below. Fix method \texttt{get} to achieve the desired functionality.
\noindent
\pythonexternalnoframe{codes/\filename b.py}
\begin{minipage}{.55\textwidth}
	\pythonexternallines{codes/\filename a.py}
\end{minipage}
\begin{minipage}{.45\textwidth}
	{\phantom{\pythonexternalnoframe{codes/\filename a.py}}}
\end{minipage}
	\clearpage
	
	%\question
\renewcommand{\filename}{20} 
You write a script \texttt{fib.py} in which you have to make two calls to the function \texttt{fib}. Expecting it to take too long, you decide to write \texttt{fib\_parallel.py}.
If $X$ is the time taken by \texttt{fib.py} to run, and $Y$ is the time taken by \texttt{fib\_parallel.py}, you find $X$ is consistently less than $Y$. Fix \texttt{fib\_parallel.py} so that $Y<X$. The method \texttt{fib\_parallel.py} has no compile-time or syntax errors in it, and implements threading correctly.
\noindent
\pythonexternalnoframe{codes/\filename b.py}
\begin{minipage}{.55\textwidth}
	\pythonexternallines{codes/\filename a.py}
\end{minipage}
\begin{minipage}{.45\textwidth}
	{\phantom{\pythonexternalnoframe{codes/\filename a.py}}}
\end{minipage}
	%\clearpage

	\question
\renewcommand{\filename}{14} 
The following script invokes the method \texttt{ui()}. A bug in it produces this error when provided a certain sequence of inputs. Assume methods \texttt{readcache} and \texttt{writecache} are appropriately defined. Fix the error. Rewrite code in the empty space on the right.
\noindent
\pythonexternalnoframe{codes/\filename b.py}
\begin{minipage}{.55\textwidth}
	\pythonexternallines{codes/\filename a.py}
\end{minipage}
\begin{minipage}{.45\textwidth}
		{\phantom{\pythonexternalnoframe{codes/\filename a.py}}}
\end{minipage}
	\clearpage
		
	% programming items
	\question
You are given a grid of \texttt{1}s and \texttt{0}s. Your job is to find out the number of \textit{clusters} of \texttt{1}s in it.\\
A \texttt{1} belongs to a \textit{cluster} if it has a \texttt{1} in either of the four positions - above, below, to its right, or to its left.
A \texttt{1} surrounded by 0s in the four positions represents a \textit{cluster} by itself.
For example, 
\begin{Verbatim}
0 0 0 1
1 0 0 0
0 1 0 1
\end{Verbatim}
has 4 \textit{clusters}, with each isolated \texttt{1} forming one cluster, while 
\begin{Verbatim}
1 1 1 1
1 0 0 0
0 1 1 1
\end{Verbatim}
has 2 \textit{clusters}.
\newline \newline
You will be given a two-dimensional matrix containing \texttt{1}s and \texttt{0}s. You are required to return an integer denoting the number of \textit{clusters} in the matrix.
\newline
\begin{Verbatim}
def find_number_of_clusters(arr):
	
\end{Verbatim}


	\newpage~\newpage
	%\question
Fill in the function \texttt{\_\_mul\_\_} in \texttt{MyFloatExt} to implement a custom-multiplication method. Different test cases have been provided at the bottom of the script. Ensure your implementation handles all of them.
\pythonexternallines{codes/22.py}
	
\end{questions}
\end{document}
	
